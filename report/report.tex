\documentclass[a4paper, 11pt]{article}
\usepackage{graphicx}
\usepackage{amsmath}
\usepackage[pdftex]{hyperref}

% Lengths and indenting
\setlength{\textwidth}{16.5cm}
\setlength{\marginparwidth}{1.5cm}
\setlength{\parindent}{0cm}
\setlength{\parskip}{0.15cm}
\setlength{\textheight}{22cm}
\setlength{\oddsidemargin}{0cm}
\setlength{\evensidemargin}{\oddsidemargin}
\setlength{\topmargin}{0cm}
\setlength{\headheight}{0cm}
\setlength{\headsep}{0cm}

\renewcommand{\familydefault}{\sfdefault}

\title{Introduction to Learning and Intelligent Systems - Spring 2015}
\author{jo@student.ethz.ch\\ stegeran@ethz.ch\\sakhadov@student.ethz.ch}
\date{\today}

\begin{document}
\maketitle

\section*{Project 4 : Classification with Missing Labels}

Training data for the last project contained very little labelled samples. Therefore we decided to use a semi-supervised learning method.

For our purposes we utilized a sci-learn python library for semi-supervised learning. Our algorithm of choice was LabelSpreading as it is more robust to noise and performed better before fine tuning parameters.

As the kernel we used Radial Basis Function(RBF). Other available option is k-Nearest Neighbours function which runs faster and does not need as much memory. However for this project the result had a higher preference than the runtime.

RBF kernel takes gamma as a parameter which we fine tuned with cross-validation.

\end{document}
