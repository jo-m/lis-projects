\documentclass[a4paper, 11pt]{article}
\usepackage{graphicx}
\usepackage{amsmath}
\usepackage[pdftex]{hyperref}

% Lengths and indenting
\setlength{\textwidth}{16.5cm}
\setlength{\marginparwidth}{1.5cm}
\setlength{\parindent}{0cm}
\setlength{\parskip}{0.15cm}
\setlength{\textheight}{22cm}
\setlength{\oddsidemargin}{0cm}
\setlength{\evensidemargin}{\oddsidemargin}
\setlength{\topmargin}{0cm}
\setlength{\headheight}{0cm}
\setlength{\headsep}{0cm}

\renewcommand{\familydefault}{\sfdefault}

\title{Introduction to Learning and Intelligent Systems - Spring 2015}
\author{jo@student.ethz.ch\\ stegeran@ethz.ch\\sakhadov@student.ethz.ch}
\date{\today}

\begin{document}
\maketitle

\section*{Project 4 : Classification with Missing Labels}

Training data for the last project contained very little labelled samples (80 out of about 40'000). Therefore we decided to use a semi-supervised learning method.

Experimenting with mixed-gaussian models did not yield very good scores: First we tried the PyMix Python GMM Library. However we did not bring that to work and failed even at the data read-in stage. Sklearn does have a GMM classifier, but it does not have semi-supervised learning builtin. Experimenting with other libraries did not yield any usable results, too.

We then decided to go an entirely different path: to use sklearns LabelPropagation and LabelSpreading implementations. This was not very successfull at the beginning, as curiously most of the predicted Y values were NaN (Not a number). Further Investigation showed that after fitting, the sklearn.LabelSpreading internal state label_distributions_ had some values set to NaN, which then propagated to the Ypred values. We choose the solution to just set every NaN to 0 in the internal state. This gave us a score of about 4. Generally we could observe that LabelSpreading gave us much better results than LabelPropagation.

The breaktrough came when we used the Sklearn LabelSpreader with a RBF (Radial basis function) kernel. This bumped our score to 0.71 immediatly.

It is to observe that LabelSpreading with RBF kernel requires quite a lot of memory and CPU power. It ran for about 10 minutes on a Macbooc Pro with 8 cores Intel i7 @ 2.30GHz. The algorithm was not runnable on a 8gb RAM machine.

Another available option is k-Nearest Neighbours function which runs faster and does not need as much memory. However for this project the result had a higher preference than the runtime.

RBF kernel takes gamma as a parameter which we fine tuned with cross-validation.

\end{document}
